\documentclass[11pt]{article}
\usepackage{geometry}                
\geometry{letterpaper}                   

\usepackage{graphicx}
\usepackage{amssymb}
\usepackage{epstopdf}
%\usepackage{natbib}
\usepackage{amssymb, amsmath}
\DeclareGraphicsRule{.tif}{png}{.png}{`convert #1 `dirname #1`/`basename #1 .tif`.png}

\title{Opinion Formation: Impacts of convincing extreme //
individuals onto a society that typically converges to one opinion}
\author{Alexander Stein, Niklas Tidbury, Elisa Wall}
\date{date} 

\begin{document}



\thispagestyle{empty}

\begin{center}
\includegraphics[width=5cm]{ETHlogo.eps}

\bigskip


\bigskip


\bigskip


\LARGE{ 	Lecture with Computer Exercises:\\ }
\LARGE{ Modelling and Simulating Social Systems with MATLAB\\}

\bigskip

\bigskip

\small{Project Report}\\

\bigskip

\bigskip

\bigskip

\bigskip


\begin{tabular}{|c|}
\hline
\\
\textbf{\LARGE{Opinion Formation: Impacts of convincing}}\\
\textbf{\LARGE{extreme individuals onto a society that typically}}\\
\textbf{\LARGE{converges to one opinion}}\\
\\
\hline
\end{tabular}
\bigskip

\bigskip

\bigskip

\LARGE{Alexander Stein, Niklas Tidbury \& Elisa Wall}



\bigskip

\bigskip

\bigskip

\bigskip

\bigskip

\bigskip

\bigskip

\bigskip

Zurich\\
December 2017\\

\end{center}



\newpage

%%%%%%%%%%%%%%%%%%%%%%%%%%%%%%%%%%%%%%%%%%%%%%%%%

\newpage
\section*{Agreement for free-download}
\bigskip


\bigskip


\large We hereby agree to make our source code for this project freely available for download from the web pages of the SOMS chair. Furthermore, we assure that all source code is written by ourselves and is not violating any copyright restrictions.

\begin{center}

\bigskip


\bigskip


\begin{tabular}{@{}p{1cm}@{}p{5cm}@{}@{}p{5cm}@{}@{}p{5cm}@{}}
\begin{minipage}{1cm}

\end{minipage}
&
\begin{minipage}{5cm}
\vspace{2mm} \large Alexander Stein

 \vspace{\baselineskip}

\end{minipage}
&
\begin{minipage}{5cm}

\large Niklas Tidbury

\end{minipage}
&
\begin{minipage}{5cm}
	
	\large Elisa Wall
	
\end{minipage}
\end{tabular}


\end{center}
\newpage

%%%%%%%%%%%%%%%%%%%%%%%%%%%%%%%%%%%%%%%



% IMPORTANT
% you MUST include the ETH declaration of originality here; it is available for download on the course website or at http://www.ethz.ch/faculty/exams/plagiarism/index_EN; it can be printed as pdf and should be filled out in handwriting


%%%%%%%%%% Table of content %%%%%%%%%%%%%%%%%

\tableofcontents

\newpage

%%%%%%%%%%%%%%%%%%%%%%%%%%%%%%%%%%%%%%%



\section{Abstract}

\section{Introduction and Motivations}
For millennia, society has consisted of many opinions and points of view. In some cases, these opinions have been oppressed, other opinions have been forced onto societies, others brainwashed. Within a democracy, these opinions are given space to spread, to change and to evolve and yet: they still converge into a general opinion. How is this possible in cases of extremism, where extreme opinions are so different compared to the majority? What effect do extreme opinions, such as that of the IS, Charles Manson and Co. have on a converging opinion of a society? We would like to examine how extreme opinions of individuals impacts such a society, and under what circumstances these opinions can have a wide-spread effect. \\*
Basing on the papers of Holme and Newman \cite{Coevolutions} such as Laguna, Abramson and Zanette \cite{Minor}, a socienty of opinions in the range [0,1] converges against an average opinion 0.5 if the parameters are set accordingly. To make it easier, we started from that situation and added the extreme opinion people in order to see how it affects the outcome. \\*
Our questions are basically the following: 
\begin{itemize}
\item What are the outcomes of n convincing individuals of extreme opinions in a society that converges to one opinion? Do we see fragmentation or polarisation of opinions?
\item What effects do these extreme opinions have in a society of low $u$ [narrow communicating interval] and $\mu$ [weight of foreign opinions]?
\item What happens when we vary the random distribution?
\end{itemize}


\section{Model and Implementation}
\subsection{Society agent in a single time step}
We have set a society of $N$ agents with opinions $x_i \in [0,1]$, $i \in \{1, N \}$. For initial condition we assume a random uniform distribution which means that all opinions are equally like.

Then we define a threshold $u$. So agents interact with each other only if their difference in opinion is smaller than this threshold. In \cite{Minor} they refer this to "bounded confidence" which corresponds to the fact that people tend to spend time with agents with similar opinion. A big u means that the society agents are open-minded and can take big changes in their opinion.

By building a threshold u into the structure, the agents only interact with similar-thinking agents, as people generally interact with the like-minded.

When two agents interact with each other they mutual adopt their opinion. This is defined by their difference of opinion and a weight $mu$ that we introduce here. \\

The concrete implementation refered to \cite{Minor} looks as follows:
\begin{equation}
\begin{aligned}
x(t+1) &= x(t) + \mu(x'(t) - x(t))  \\
x'(t+1) &= x(t) + \mu(x(t) - x'(t)) 
\end{aligned}
\end{equation}

Note that only weight adjustment $0 < \mu < 1$ are allowed to guarantee that the opinions after are between 0 and 1. Next to this a negative value would mean that opinion of the two agents split even more and value bigger than 1 would mean that updated opinion of x will be closer to the opinion of the prior opinion of x' than the updated opinion of x'. 

Also note that using this model, close opinions come even closer and opinions further away from other opinions do not change. "Close" and "far-away" are characterized in $u$.

Until now we have defined how two agents interact with each other and under which threshold they meet. Now we define a single time step in this way that every agent choose randomly another agent of the society within the threshold to interact with him as described above. Further we iterate over the $T$ time steps to get the final update of opinion.

According to \cite{Minor}: If we take reasonable values for $u$ and$\mu$ , this typically follows to a convergence of a unique opinion around $x$ = 0.5.


\subsection{Extremists}
After that, we insert the extreme Opinion Individuals to our society. The parameters for each of the two extreme sides are the following: The number of extreme Opinion Indivuduals $n$, the number of society members that can be influenced within a timestep $p$, the probability of persuasion $\kappa$ and the threshold of interaction $infop$. \\*
So within that probability, the extreme opinion individuals can convince somebody of either opinion 1 or 0.

\paragraph{The Extreme Opinions}
Within this situation we add $n$ extreme individuals in the beginning with a non-changing opinion in the outer parts of the Gaussian curve at 0 and 1. We assume that the agents are charismatic and have good rhetorics in order to convince $p$ other agents by a probability of success of $\kappa$. This can be interpreted as the influence of certain individuals can be higher, such as social media stars or politicians with extreme opinions. Again, we assume this influence only withing a threshold.

\subsection{Effective values}
Describe here: What are the effective values? \\
$n_{eff} = n*p$ follows to: it only matters the product of n and p and not the single values itself.

$\kappa*n_{eff}$ matters but not kappa and neff itself

Finally we will have two effective parameters: infop and $\kappa*n_{eff}$

\section{A world without extremists}
Insert a description of the effects that are already mentioned in the paper.

\subsection{The weigh mu gives the speed of convergence}
insert the plots and describe the effect


\subsection{The threshold u defines the cluster building}
According to \cite{Minor}, such a society of N agents with random uniform distributed opinions $x_i$ in [0,1] will converge to a common opinion around 0.5 if the comunicating interval $u$ of agents communicating with each other is big enough, more precise if $u>0.3$.

insert the plot and describe the effect

\section{A world with extremists}
We extended the paper. The experiments and their results should be discussed here.


\section{Discussion}
a typical discussion... what was good/bad? Problems? Interpretation of our simulation results? Limits of the model?

\section{Summary and Outlook}
Until now we considered only symmetric random distributions and symmetric distribution of extremists. What is gonna happen if this will be asymmetric?


If we do no Gaussian distribtion: a normal distribution is a sensible alternative opinion distribution where 0.5 is the main stream opinion and other opinions are spreaded around the main stream distribution. Sigma would be measure for "how far" the other opinions are spreaded from the main stream opinion.


%\section{References}
\begin{thebibliography}{99}
\bibitem{Coevolutions} Peter Holme and M. E. J. Newman. \textit{Nonequilibrium phase transition in the coevolution of networks and opinions}. arXiv:physics/0603023v3, 9 March 2006.

\bibitem{Minor} M. F. Laguna, Guillermo Abramson, and Damian H. Zanette. \textit{Minorities in a Model for Opinion Formation}. Wiley Periodicals, Inc., Vol. 9, No.4, 5 January 2004

\end{thebibliography} 


\end{document}
